% LaTeX Template For MATH 490 @ VCU
\documentclass[11pt]{article}

\usepackage{hyperref}
\usepackage{amsmath}
\usepackage{amsthm}
\usepackage{amssymb}
\usepackage{enumerate}
\usepackage{enumitem}
\usepackage{titlesec}
\usepackage{multicol}
\usepackage{multirow}
\usepackage{mathtools}
\usepackage{mdframed}
\usepackage{tocloft}
\usepackage{tcolorbox}
\usepackage{extarrows}

\setlist{nosep}
% \setlist[enumerate]{label=(\alph*)}

\renewcommand{\arraystretch}{0.75}

\definecolor{defcolor}{RGB}{255,236,236}    % light red
\definecolor{ngtcolor}{RGB}{255,242,242}    % lighter red
\definecolor{lnkcolor}{RGB}{0,0,180}        % blue
\definecolor{thmcolor}{RGB}{236,236,255}    % light blue
\definecolor{lemcolor}{RGB}{239,239,255}    % lighter blue
\definecolor{procolor}{RGB}{242,242,255}    % lighter lighter blue
\definecolor{crlcolor}{RGB}{245,245,255}    % lighter lighter lighter blue
\definecolor{xmpcolor}{RGB}{255,240,225}    % light orange
\definecolor{rmkcolor}{RGB}{233,255,235}    % light green
\definecolor{axicolor}{RGB}{255,255,233}    % light yellow
\definecolor{notcolor}{RGB}{255,255,244}    % lighter yellow
\definecolor{whacolor}{RGB}{250,250,250}    % lighter gray
\definecolor{reccolor}{RGB}{255,244,255}    % lighter purple

\hypersetup{
    colorlinks,
    citecolor=lnkcolor,
    filecolor=lnkcolor,
    linkcolor=lnkcolor,
    urlcolor=lnkcolor
}

\newtheoremstyle{break}
    {\topsep/1.5} % space above
    {\topsep/2.2} % space below
    {}          % body font
    {}          % indent amount
    {\rmfamily} % theorem head font
    {.}          % punctuation after theorem head
    {\newline}  % space after theorem head
    {\textbf{\thmname{#1}\thmnumber{ #2}}\thmnote{\text{ (#3)}}}
                % theorem hed spec. (empty = "normal")

\newtheoremstyle{no_label}
    {\topsep/1.5} % space above
    {\topsep/2.2} % space below
    {}          % body font
    {}          % indent amount
    {\rmfamily} % theorem head font
    {.}          % punctuation after theorem head
    {\newline}  % space after theorem head
    {\textbf{\thmname{#1}\thmnumber{}}\thmnote{\text{ (#3)}}}
                % theorem hed spec. (empty = "normal")

\theoremstyle{break}
\newmdtheoremenv[
    backgroundcolor=thmcolor,
    linecolor=black,
    linewidth=1pt,
    topline=true,
    bottomline=true,
    rightline=true,
    skipabove=\topsep/1.5,
    skipbelow=\topsep/2.2
]{theorem}{Theorem}[section]
\newmdtheoremenv[
    backgroundcolor=crlcolor,
    linecolor=black,
    linewidth=1pt,
    topline=true,
    bottomline=true,
    rightline=true,
    skipabove=\topsep/1.5,
    skipbelow=\topsep/2.2
]{corollary}[theorem]{Corollary}
\newmdtheoremenv[
    backgroundcolor=lemcolor,
    linecolor=black,
    linewidth=1pt,
    topline=true,
    bottomline=true,
    rightline=true,
    skipabove=\topsep/1.5,
    skipbelow=\topsep/2.2
]{lemma}[theorem]{Lemma}
\newmdtheoremenv[
    backgroundcolor=axicolor,
    linecolor=black,
    linewidth=1pt,
    topline=true,
    bottomline=true,
    rightline=true,
    skipabove=\topsep/1.5,
    skipbelow=\topsep/2.2
]{axiom}[theorem]{Axiom}
\newmdtheoremenv[
    backgroundcolor=procolor,
    linecolor=black,
    linewidth=1pt,
    topline=true,
    bottomline=true,
    rightline=true,
    skipabove=\topsep/1.5,
    skipbelow=\topsep/2.2
]{proposition}[theorem]{Proposition}
\newmdtheoremenv[
    backgroundcolor=notcolor,
    linecolor=black,
    linewidth=1pt,
    topline=true,
    bottomline=true,
    rightline=true,
    skipabove=\topsep/1.5,
    skipbelow=\topsep/2.2
]{notation}[theorem]{Notation}
\newmdtheoremenv[
    backgroundcolor=defcolor,
    linecolor=black,
    linewidth=1pt,
    topline=true,
    bottomline=true,
    rightline=true,
    skipabove=\topsep/1.5,
    skipbelow=\topsep/2.2
]{definition}[theorem]{Definition}
\newmdtheoremenv[
    backgroundcolor=ngtcolor,
    linecolor=black,
    linewidth=1pt,
    topline=true,
    bottomline=true,
    rightline=true,
    skipabove=\topsep/1.5,
    skipbelow=\topsep/2.2
]{negation}[theorem]{Negation}
\newmdtheoremenv[
    backgroundcolor=rmkcolor,
    linecolor=black,
    linewidth=1pt,
    topline=true,
    bottomline=true,
    rightline=true,
    skipabove=\topsep/1.5,
    skipbelow=\topsep/2.2
]{remark}[theorem]{Remark}
\newmdtheoremenv[
    backgroundcolor=xmpcolor,
    linecolor=black,
    linewidth=1pt,
    topline=true,
    bottomline=true,
    rightline=true,
    skipabove=\topsep/1.5,
    skipbelow=\topsep/2.2
]{example}[theorem]{Example}
\newmdtheoremenv[
    backgroundcolor=whacolor,
    linecolor=black,
    linewidth=1pt,
    topline=true,
    bottomline=true,
    rightline=true,
    skipabove=\topsep/1.5,
    skipbelow=\topsep/2.2
]{problem}[theorem]{Problem}
\newmdtheoremenv[
    backgroundcolor=whacolor,
    linecolor=black,
    linewidth=1pt,
    topline=true,
    bottomline=true,
    rightline=true,
    skipabove=\topsep/1.5,
    skipbelow=\topsep/2.2
]{exercise}[theorem]{Exercise}

\theoremstyle{no_label}
\newmdtheoremenv[
    backgroundcolor=whacolor,
    linecolor=black,
    linewidth=1pt,
    topline=true,
    bottomline=true,
    rightline=true,
    skipabove=\topsep/1.5,
    skipbelow=\topsep/2.2
]{question}{Question}
\newmdtheoremenv[
    backgroundcolor=reccolor,
    linecolor=black,
    linewidth=1pt,
    topline=true,
    bottomline=true,
    rightline=true,
    skipabove=\topsep/1.5,
    skipbelow=\topsep/2.2
]{recall}{Recall}

\DeclareMathOperator{\arcsec}{arcsec}
\DeclareMathOperator{\arccot}{arccot}
\DeclareMathOperator{\arccsc}{arccsc}
\DeclareMathOperator{\interior}{int}
\DeclareMathOperator{\closure}{cl}
\DeclareMathOperator{\boundary}{bd}

\newcommand{\differentiate}[1]{\dfrac{\dd}{\dd{#1}}}
\newcommand{\pdifferentiate}[1]{\dfrac{\partial}{\partial {#1}}}
\newcommand{\derivative}[2]{\dfrac{\dd{#1}}{\dd{#2}}}
\newcommand{\scndderivative}{D^2\!\,}
\newcommand{\highderivative}[1]{D^{#1}\!\,}
\newcommand{\dirderivative}[1]{D_{#1}\!\,}
\newcommand{\pderivative}[2]{\dfrac{\partial {#1}}{\partial {#2}}}
\newcommand{\scndpderivative}[3]{\dfrac{\partial^2 {#1}}{\partial {#3}\partial {#2}}}
\newcommand{\highpderivative}[4]{\dfrac{\partial^{#2} {#1}}{\partial{#4}\cdots\partial{#3}}}
\newcommand{\dd}{\text{d}}
\newcommand{\ddi}{\text{$\,$d}}
\newcommand{\qqed}{{\hfill$\blacksquare$}}
\newcommand{\defeq}{\overset{\text{def}}{=}}
\newcommand{\transpose}{\text{T}}
\newcommand{\bbR}{\mathbb{R}}
\newcommand{\bbN}{\mathbb{N}}
\newcommand{\calL}{\mathcal{L}}
\newcommand{\bfa}{\textbf{a}}
\newcommand{\bfc}{\textbf{c}}
\newcommand{\bfe}{\textbf{e}}
\newcommand{\bff}{\textbf{f}}
\newcommand{\bfg}{\textbf{g}}
\newcommand{\bfh}{\textbf{h}}
\newcommand{\bfp}{\textbf{p}}
\newcommand{\bfr}{\textbf{r}}
\newcommand{\bfv}{\textbf{v}}
\newcommand{\bfu}{\textbf{u}}
\newcommand{\bfx}{\textbf{x}}
\newcommand{\bfy}{\textbf{y}}


\linespread{2}
\setlength{\textwidth}{6.9in}
\setlength{\textheight}{9.2in}
\setlength{\oddsidemargin}{-0.2in}
\setlength{\evensidemargin}{-0.2in}
\setlength{\topmargin}{-0.2in}
\setlength{\headheight}{0in}
\setlength{\headsep}{0in}
\setlength{\footskip}{0.5in}
\setlength{\multicolsep}{6.2pt}
\setlength{\belowdisplayskip}{0pt}
%\setlength{\belowdisplayshortskip}{0pt}
\setlength{\abovedisplayskip}{0pt}
%\setlength{\abovedisplayshortskip}{0pt}

\setcounter{section}{4}
\numberwithin{equation}{theorem}

\makeatletter
\newcommand{\vast}{\bBigg@{4}}
\newcommand{\Vast}{\bBigg@{5}}
\makeatother

\newcommand*\samethanks[1][\value{footnote}]{\footnotemark[#1]}

\title{\textbf{Calculus A (II) One-to-One Tutoring}\\ \Large Introduction to Integrals}
\author{Chang, Yung-Hsuan}

\begin{document}
\maketitle

\begin{definition}[Antiderivative]
    For a given function $f(x)$, the function $F(x)$ is called an antiderivative of $f$ if $\derivative{F}{x}(x)=f(x)$.
\end{definition}

\begin{example}
    Find an antiderivative of the following functions: \vspace{-1em}
    \begin{multicols}{4}
        \begin{enumerate}
            \item $0$;
            \item $1$;
            \item $x$;
            \item $x^3$;
            \item $x^n$, $n\in\bbN$;
            \item $\dfrac{1}{x}$;
            \item $\sin x$;
            \item $\cos x$;
            \item $e^x$;
            \item $\sin 2x$;
            \item $\cos 2x$;
            \item $e^{2x}$;
            \item $\sin kx$;
            \item $\cos \ell x$;
            \item $e^{rx}$; and
            \item $a^x$.
        \end{enumerate}
    \end{multicols}
\end{example}


\begin{remark}
    We use the long crowbar $\displaystyle\int$ and the infinitesimal $\dd x$ to clamp the integrand (usually, a function $f(x)$ in this case) $\displaystyle\int f(x)\ddi x$ to indicate the ``family'' of antiderivatives of $f(x)$. Then we call the family of antiderivatives to be ``(indefinite) integrals.'' It is called indefinite since the difference between each member in the family is a constant but not deemed to be zero.
\end{remark}

\begin{remark}
    If the crowbar in an integral is with subscript and superscript, then the integral is called a ``definite integral.''
\end{remark}

\begin{theorem}[Fundamental Theorem of Calculus]
    Let $f$ be a continuous function on an interval $[a, b]$ and let $F$ be an antiderivative of $f$. Then
    \begin{enumerate}
        \item $\derivative{F}{x}(x)=f(x)$; and
        \item $\displaystyle\int_a^b f(x)\ddi x=F(b)-F(a)$.
    \end{enumerate}
\end{theorem}

\begin{example}
    Evaluate the following indefinite integrals:\vspace{-1.8em}
    \begin{multicols}{3}
        \begin{enumerate}
            \item $\displaystyle\int 2x\ddi x$;
            \item $\displaystyle\int x\ddi x$;
            \item $\displaystyle\int 4x^3\ddi x$;
            \item $\displaystyle\int x^3\ddi x$;
            \item $\displaystyle\int (n+1)\cdot x^n\ddi x$, $n\in\bbN$;
            \item $\displaystyle\int x^n\ddi x$, $n\in\bbN$;
            \item $\displaystyle\int \dfrac{1}{x}\ddi x$;
            \item $\displaystyle\int \sin x\ddi x$;
            \item $\displaystyle\int \cos x\ddi x$;
            \item $\displaystyle\int e^x\ddi x$;
            \item $\displaystyle\int \sin 2x\ddi x$;
            \item $\displaystyle\int \cos 2x\ddi x$;
            \item $\displaystyle\int e^{2x}\ddi x$;
            \item $\displaystyle\int \sin kx\ddi x$; and
            \item $\displaystyle\int \cos \ell x\ddi x$.
        \end{enumerate}
    \end{multicols}\vspace{0.1em}
\end{example}

\begin{exercise}
    Use Example 3.6, evaluate the following definite integrals:\vspace{-1.8em}
    \begin{multicols}{3}
        \begin{enumerate}
            \item $\displaystyle\int_0^{\sqrt{2}} x\ddi x$;
            \item $\displaystyle\int_0^{\sqrt{2}} x^3\ddi x$;
            \item $\displaystyle\int_1^2 \dfrac{1}{x}\ddi x$;
            \item $\displaystyle\int_0^\pi \sin x\ddi x$;
            \item $\displaystyle\int_0^\pi \cos x\ddi x$;
            \item $\displaystyle\int_0^1 e^x\ddi x$;
            \item $\displaystyle\int_0^\pi \sin 2x\ddi x$;
            \item $\displaystyle\int_0^\pi \cos 2x\ddi x$; and
            \item $\displaystyle\int_0^{\frac{1}{2}} e^{2x}\ddi x$.
        \end{enumerate}
    \end{multicols}\vspace{0.1em}
\end{exercise}



\begin{theorem}[Properties of Definite Integrals]
    Let $f, g$ be integrable on $[a, b]$. Then the following are true:
    \begin{enumerate}
        \item $\displaystyle\int_a^b f(x)\ddi x=-\int_b^a f(x)\ddi x$;
        \item $\displaystyle\int_a^b f(x)\ddi x=\int_a^c f(x)\ddi x+\int_c^b f(x)\ddi x$ for $c\in[a, b]$;
        \item $\displaystyle\int_a^b \alpha f(x)+g(x)\ddi x=\alpha\int_a^b f(x)\ddi x+\int_a^b g(x)\ddi x$;
        \item if $f(x)\geq g(x)$ for all $x\in[a, b]$, then $$\int_a^b f(x)\ddi x\geq\int_a^b g(x)\ddi x;$$
        \item $|f|$ is integrable;
        \item $\displaystyle\left|\int_a^b f(x)\ddi x\right|\leq\int_a^b |f(x)|\ddi x$; and
        \item $fg$ is integrable.
    \end{enumerate}
\end{theorem}

\begin{theorem}[Substitution Rule]
    Suppose that $f$ is continuous on $[c, d]\supseteq[a, b]$ and $\derivative{\phi}{t}(t)$ are conitnuous in $[\alpha, \beta]$ with $\phi(t)\in[c, d]$ for all $t\in[\alpha, \beta]$, where $\alpha=\phi(a)$ and $\beta=\phi(b)$. Then $$\int_a^b f(x)\ddi x=\int_{\alpha}^{\beta}f(\phi(t))\derivative{\phi}{t}(t)\ddi t.$$
\end{theorem}

\begin{remark}
    The substitution rule comes from the chain rule.
\end{remark}

\begin{example}
    Evaluate $\displaystyle\int\dfrac{2x+1}{x^2+x+1}\ddi x$.
\end{example}


\begin{example}
    Evaluate $\displaystyle\int x e^{x^2}\ddi x$.
\end{example}


\begin{example}
    Evaluate $\displaystyle\int\dfrac{\sin\sqrt{x}}{\sqrt{x}}\ddi x$.
\end{example}


\begin{example}
    Evaluate $\displaystyle\int\dfrac{e^x}{1+e^{2x}}\ddi x$.
\end{example}


\begin{example}
    Evaluate $\displaystyle\int\dfrac{\ln x}{x}\ddi x$.
\end{example}


\begin{theorem}[Integration by Parts]
    If the functions $f$ and $g$ are differentiable and their derivatives $\derivative{f}{x}$ and $\derivative{g}{x}$ are continuous on $[a, b]$, then $$\int_a^b f(x)\derivative{g}{x}(x)\ddi x=\left[f(x)g(x)\right]_a^b-\int_a^b g(x)\derivative{f}{x}(x)\ddi x.$$
\end{theorem}

\begin{remark}[DI Method]
    For convenience, the DI method is much quicker than think about $f$ and $g$. Choose the most-easily-integrable function $h_1$ to be under I, and the other $h_2$ to be under D. Then use the following table to compute.
    
    \begin{center}
        \begin{tabular}{c c c}
            & D & I \\
            $+$ & $h_2(x)$ & $h_1(x)$ \\
            $-$ & $\derivative{h_2}{x}(x)$ & $\displaystyle\int h_1(x)\ddi x$ \\
            $+$ & $\derivative{^2h_2}{x^2}(x)$ & $\displaystyle\iint h_1(x)\ddi x\dd x$ \\
            $-$ & $\derivative{^3h_2}{x^3}(x)$ & $\displaystyle\iiint h_1(x)\ddi x\dd x\dd x$ \\
            & $\vdots$ & $\vdots$ \\
            $(-1)^{k}$ & $\derivative{^kh_2}{x^k}(x)$ & $\displaystyle\overbrace{\idotsint}^k h_1(x)\overbrace{\ddi x\cdots\dd x}^k$ \\
        \end{tabular}
    \end{center}
    Then \begin{align*}
        \int h_2(x)h_1(x)\ddi x&=h_2(x)\cdot\int h_1(x)\ddi x\\
        &\quad-\derivative{h_2}{x}(x)\cdot\iint h_1(x)\ddi x\\
        &\quad+\derivative{^2h_2}{x^2}(x)\cdot\iiint h_1(x)\ddi x\dd x\\
        &\quad-\cdots\\
        &\quad+(-1)^k\int\derivative{^kh_2}{x^k}(x)\cdot\overbrace{\idotsint}^k h_1(x)\overbrace{\ddi x\cdots\dd x}^k\ddi x.
    \end{align*}
\end{remark}

\begin{example}
    Evaluate $\displaystyle\int\ln x\ddi x$.
\end{example}


\begin{example}
    Evaluate $\displaystyle\int x^2e^x\ddi x$.
\end{example}


\begin{example}
    Evaluate $\displaystyle\int e^x\sin x\ddi x$.
\end{example}


\begin{theorem}[Iterated Integral]
    If $f(x, y)$ is bounded on the rectangle $K=\{(x, y)\in\bbR^2\mid x\in[a, b]\text{\ and\ }y\in[c,d ]\}$ and $$F(y)\coloneq\int_a^b f(x, y)\ddi x$$ is well-defined, then $$\int_K f(x, y)\ddi A=\int_c^d\left(\int_a^b f(x, y)\ddi x\right)\dd y.$$
\end{theorem}

\begin{example}
    Evaluate the following integrals:
    \begin{enumerate}
        \item $\displaystyle\iint_{K_1}\sin x\cos y\ddi x\ddi y$, $K_1=[0, \frac{\pi}{3}]\times[0, \frac{\pi}{6}]$;
        \item $\displaystyle\iint_{K_2}e^x\sin y\ddi x\ddi y$, $K_2=[0, 1]\times[0, \alpha]$;
        \item $\displaystyle\iint_{K_3}x^{2n-1}y^me^{x^ny^{m+1}}\ddi x\ddi y$; and
        \item $\displaystyle\iint_{K_4}\dfrac{y^2}{x^2y^2+1}\ddi x\ddi y$, $K_4=[0, 1]\times[0, 1]$.
    \end{enumerate}
\end{example}


\begin{example}
    Let $D=\{(x, y)\in\bbR^2\mid x\geq 0, y\geq 0,\text{\ and\ }x+y\leq 2\}$. Evaluate $$\iint_D x+y\ddi x\ddi y.$$
\end{example}

\end{document}